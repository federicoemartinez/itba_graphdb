% This is LLNCS.DEM the demonstration file of
% the LaTeX macro package from Springer-Verlag
% for Lecture Notes in Computer Science,
% version 2.4 for LaTeX2e as of 16. April 2010
%

\documentclass{llncs}
\usepackage[ansinew,utf8]{inputenc}
%
\usepackage{makeidx}  % allows for indexgeneration
%
\begin{document}
%
\frontmatter          % for the preliminaries
%
\pagestyle{headings}  % switches on printing of running heads
\addtocmark{Bases de datos de grafos con manejo de datos espaciales} % additional mark in the TOC
%
\chapter*{Preface}
%
TODO: completar
¿Qué ponemos acá?

\vspace{1cm}
\begin{flushright}\noindent
April 1995\hfill Walter Olthoff\\
Program Chair\\
ECOOP'95
\end{flushright}
%

%
\tableofcontents
%
\mainmatter              % start of the contributions
%
\title{Bases de datos de grafos con manejo de datos espaciales.
Un análisis comparativo}
%
\titlerunning{Bases de datos de grafos con manejo de datos espaciales}  % abbreviated title (for running head)
%                                     also used for the TOC unless
%                                     \toctitle is used
%
\author{Federico Martinez\inst{1} \and Ariel Aizemberg\inst{2}}
%
\authorrunning{Federico Martinez et al.} % abbreviated author list (for running head)
%
%%%% list of authors for the TOC (use if author list has to be modified)
\tocauthor{Federico Martinez, Ariel Aizemberg}
%
\institute{Universidad Nacional de Quilmes, Roque Sáenz Peña 352, Bernal, Buenos Aires, Argentina,\\
\email{federico.martinez@unq.edu.ar}
\and
Instituto Tecnologico Buenos Aires,
Av. Madero 399, Buenos Aires, Argentina,\\
\email{aaizemberg@itba.edu.ar}}

\maketitle              % typeset the title of the contribution

\begin{abstract} The abstract should summarize the contents of the
paper using at least 70 and at most 150 words. It will be set in
9-point font size and be inset 1.0 cm from the right and left margins.
There will be two blank lines before and after the Abstract. \dots
\keywords{computational geometry, graph theory, Hamilton cycles}
\end{abstract} % 

\section{Bases de datos de grafos con manejo de datos espaciales}
\subsection{Introducción} % 
Las bases de datos NoSQL han
adquirido una gran popularidad en el último tiempo. Entre los
distintos tipos que existen, las bases de datos de grafos se destacan
por su capacidad de almacenar elementos de características diferentes,
no estructuradas y fundamentalmente, las relaciones que existen entre
ellos.

En la actualidad existen varias bases de datos de grafos, tanto open
source, como de código propietario. Son varias las que introducen la
posibilidad de manejar además datos espaciales o georreferenciados.
Entre ellas podemos citar: Neo4j\footnote{http://neo4j.com/}, ArangoDB\footnote{http://www.arangodb.com/}
, OrientDB\footnote{http://www.orientechnologies.com/orientdb/}, Titan\footnote{http://thinkaurelius.github.io/titan/}; las
cuales permiten en distinta medida almacenar y consultar datos
espaciales. Las bases de datos de grafos son muy útiles cuando el
problema que se quiere resolver tiene distintos tipos de entidades
relacionadas entre sí.  Son eficientes para resolver problemas de
búsquedas en grafos, pero sabemos que los datos almacenados suelen
contener información espacial o geográfica. Entonces, nos preguntamos
¿que pasa cuando incorporamos la componente espacial en nuestras
consultas?

Elegimos a Neo4j y ArangoDB ya que estas poseen manejo de datos
geográficos sin necesidad de motores de indexado externos para
realizar una comparación. Un ejemplo típico es una red social. En ella
los usuarios son nodos en un grafo y la “amistad” entre dos usuarios
es un eje que los relaciona. Estos nodos pueden tener propiedades como
el nombre, edad, género, etc. Las fotos subidas por los usuarios
pueden ser también un nodo, con otro conjunto de propiedades: url,
fecha de publicación, etc. En este caso, pueden existir relaciones
entre usuarios y fotos con distinta semántica: le gusta la imagen,
comparte la imagen, está etiquetado en la imagen, etc. Cada una de
estas relaciones puede modelarse con un eje. En este escenario,
podemos pensar que una consulta típica es: ¿Cuáles son los amigos de
mis amigos?

Existen problemas donde estos nodos se encuentran distribuidos en el
espacio, por ejemplo los nodos pueden representar ciudades y los ejes
autopistas, rutas o calles entre ellos. Para estos casos es
interesante contar con la capacidad de realizar consultas sobre la
georeferenciación de los datos y no solo sobre sus relaciones, a fin
de poder responder, por ejemplo, cual es el camino más corto entre dos
ciudades.

\subsection{Metodología}
Para analizar las capacidades de manejo de datos geográficos que presentan ambas bases de datos decidimos utilizar un conjunto de datos proveniente del sitio openflights\footnote{ttp://openflights.org/data.html}. El mismo presenta información de 6977 aeropuertos, los cuales tienen, entre otros atributos, latitud y longitud. Además hay 5888 aerolíneas y 59036 rutas. Las rutas van de un aeropuerto a otro y son gestionadas por una aerolínea.

\begin{itemize}
\item Cada aeropuerto es un nodo, al igual que las aerolíneas y las rutas.
\item Un eje entre una ruta y una aerolínea indica que ésta es responsable de la ruta.
\item Las rutas pueden relacionarse con los aeropuertos mediante dos tipos distintos de ejes: Origen y Destino, los cuales indican, respectivamente, si el aeropuerto es donde la ruta comienza o termina.
\end{itemize}

Para el modelo de datos generado, realizamos las siguientes consultas:
\begin{enumerate}
\item Aeropuerto a menos de 100 kilómetros de la Ciudad de Buenos Aires.
\item Aeropuertos a 100 kilómetros de la Ciudad de Buenos Aires que permiten viajar a Barajas.
\item Itinerarios que parten de un aeropuerto a menos de 400 kilómetros de la Ciudad de Buenos Aires, que tengan una escala y terminan en Barajas.
\item Itinerarios que parten de un aeropuerto a menos de 400 km de la Ciudad de Buenos Aires, tienen una escala y terminan a menos de 400 km de Chipre.
\end{enumerate}

\subsection{Resultados}
Para cada consulta consideramos el tiempo de ejecución hasta obtener los resultados. Analizamos también el tiempo necesario para la carga inicial y el espacio necesario en disco para cada una de las bases. La carga de datos y los experimentos se ejecutaron sobre una máquina virtual, ésta dispone de 3 GB de RAM y 2 cores. La máquina física es un Core i5 4690 de 3.5 GHz con 16 GB de RAM.
\begin{table}
\caption{Resultados de los experimentos}
\begin{center}
\begin{tabular}{r@{\quad}ll}%{r@{\quad}rll}
\hline
\multicolumn{1}{l}{Experimento}&\multicolumn{1}{l}{Neo4J}&\multicolumn{1}{l}{ArangoDB}\\
\hline\rule{0pt}{12pt}
Carga & 20 minutos 27 segundos & 25 minutos 37 segundos \\
Espacio en disco & 165.9 MB & 437.3 MB \\
Consulta 1 & 0.063 segundos & 0.012 segundos \\
Consulta 2 & 0.016 segundos & 0.052 segundos \\
Consulta 3 & 0.227 segundos & 3.697 segundos \\
Consulta 4 & 2.319 segundos & 4.901 segundos \\
\hline
\end{tabular}
\end{center}
\end{table}

\subsection{Conclusiones}

Si bien ambas bases de datos proveen soporte para el manejo de datos geográficos y las dos funcionan muy bien para responder consultas sencillas, Neo4j logra imponerse como una mejor alternativa cuando las consultas requieren recorrer el grafo o funcionalidades más complejas como importar mapas.
Si bien Arango logra resolver correctamente las consultas sencillas (por ejemplo la consulta 1, que no requería recorrer las relaciones del grafo, es resuelta más rápidamente por Arango), no está a la altura del soporte que brinda Neo4j.
Las principales falencias que encontramos en ArangoDB son:
Su lenguaje de consulta no parece estar pensado para recorrer grafos. Parece haber sido diseñado para trabajar con documentos y luego se le agregaron casi de manera ad-hoc nociones de grafos. Esto genera que escribir queries que combinen información espacial y recorrer el grafo sea ineficiente además de complejo.
Si bien es muy bueno que la base brinde soporte geográfico nativo, el mismo es muy básico. Por esa razón es imposible escribir consultas que requieren de capacidades más complejas.
 
El informe completo, las tablas, graficos y el código fuente del presente estudio, se puede consultar en la siguiente URL: 

\textit{https://github.com/federicoemartinez/itba\_graphdb}


%
% ---- Bibliography ----
%
\begin{thebibliography}{5}
%
\bibitem {clar:eke}
Clarke, F., Ekeland, I.:
Nonlinear oscillations and
boundary-value problems for Hamiltonian systems.
Arch. Rat. Mech. Anal. 78, 315--333 (1982)

\bibitem {clar:eke:2}
Clarke, F., Ekeland, I.:
Solutions p\'{e}riodiques, du
p\'{e}riode donn\'{e}e, des \'{e}quations hamiltoniennes.
Note CRAS Paris 287, 1013--1015 (1978)

\bibitem {mich:tar}
Michalek, R., Tarantello, G.:
Subharmonic solutions with prescribed minimal
period for nonautonomous Hamiltonian systems.
J. Diff. Eq. 72, 28--55 (1988)

\bibitem {tar}
Tarantello, G.:
Subharmonic solutions for Hamiltonian
systems via a $\bbbz_{p}$ pseudoindex theory.
Annali di Matematica Pura (to appear)

\bibitem {rab}
Rabinowitz, P.:
On subharmonic solutions of a Hamiltonian system.
Comm. Pure Appl. Math. 33, 609--633 (1980)

\end{thebibliography}
\end{document}
